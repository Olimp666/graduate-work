% !TEX root =..\main.tex
\section*{ВВЕДЕНИЕ}\addcontentsline{toc}{section}{ВВЕДЕНИЕ}
    В последние годы в игровой индустрии наблюдается устойчивый рост интереса к естественно-речевым интерфейсам, которые делают взаимодействие 
    пользователя с виртуальным миром более интуитивным и доступным. Особенно перспективно применение голосового управления в жанре стратегии 
    реального времени (RTS): высокие темпы игры, многопользовательский режим и сложный микроменеджмент требуют от игрока быстрого и удобного 
    способа отдачи команд. Одновременно растут требования к кроссплатформенности и расширяемости игровых проектов, чему традиционные 
    узкоспециализированные движки зачастую не отвечают.

    Выпускная квалификационная работа посвящена разработке сетевой RTS-игры на языке C\# с интегрированным голосовым интерфейсом. В качестве основы 
    выбран свободный и кроссплатформенный движок Godot Engine, чья модульная архитектура и поддержка Mono позволяют унаследовать большую часть 
    логики существующего кликнта на Citrus Engine, одновременно решив его фундаментальные ограничения.

    Научная новизна исследования заключается в комплексной интеграции системы автоматического распознавания речи (ASR) и синтеза речи (TTS) в 
    многопользовательскую RTS, построенной на Godot, а практическая значимость -- в создании масштабируемой архитектуры, пригодной для учебных и 
    демонстрационных целей, а впоследствии — для коммерческих проектов.

    Работа состоит из трёх разделов. В первой главе «Постановка задачи» обоснованы актуальность темы, сформулированы цель и задачи 
    работы, приведён обзор исходного кликнта и содержится обоснование выбора Godot. Вторая глава «Теоретический анализ» 
    содержит сравнительный обзор игровых движков, сетевых архитектур и ASR/TTS-решений. 
    Третья глава «Разработка проекта» описывает практическую реализацию клиент-серверной RTS с голосовым управлением: 
    настройку окружения, перенос логики на новый движок, дополнение серверной логики. 
    В заключении приведены основные результаты и направления дальнейшего развития.