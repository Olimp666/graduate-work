% !TEX root =..\main.tex
\section*{ТЕРМИНЫ, ОПРЕДЕЛЕНИЯ И СОКРАЩЕНИЯ}\addcontentsline{toc}{section}{ТЕРМИНЫ, ОПРЕДЕЛЕНИЯ И СОКРАЩЕНИЯ}
В настоящей письменной работе применены следующие термины с
соответствующими определениями.

\textbf{Игровой движок} -- программная платформа, обеспечивающая отрисовку графики, обработку ввода, физику и другие базовые подсистемы для создания игр.

\textbf{Библиотека} -- набор готовых функций, классов и ресурсов, предназначенных для повторного использования в разных проектах.

\textbf{RTS} -- жанр стратегии в реальном времени, в котором игроки одновременно управляют базой, ресурсами и юнитами.

\textbf{Туман войны} -- механизм скрытия частей карты, недоступных прямому обзору игрока, обычно представленный затемнением или неразличимой текстурой.

\textbf{Юнит} -- игровой объект (единица), управляемый игроком или ИИ, например боевая единица в стратегии.

\textbf{ASR} -- система автоматического распознавания речи, преобразующая аудиосигнал в текст.

\textbf{TTS} -- система синтеза речи, генерирующая звуковую дорожку на основе текстовых данных.

\textbf{SDK} -- комплект средств разработки, включающий библиотеки, документацию и инструменты для создания ПО.

\textbf{HUD} -- интерфейс на экране (англ. Heads-Up Display), отображающий важные игровые параметры: здоровье, ресурсы, миникарту и т. п.

\textbf{gRPC} -- фреймворк для высокопроизводительного удалённого вызова процедур по протоколу HTTP/2.

\textbf{UDP} -- протокол пользовательских датаграмм без установления соединения и гарантии доставки, обеспечивающий низкую задержку.

\textbf{Клиент} -- компонент сети или приложения, отправляющий запросы серверу и получающий от него данные.

\textbf{Сервер} -- компонент сети или приложения, обрабатывающий запросы клиентов, хранящий данные и управляющий логикой.

\textbf{Кроссплатформенность} -- способность программного продукта работать на разных ОС и устройствах без существенных изменений кода.

\textbf{Сцена} -- в Godot структурированный набор узлов, сохраняемый в файле и инстанцируемый как единый объект.

\textbf{Узел} -- базовый строительный блок сцены в Godot, обладающий свойствами, методами и способностью быть частью иерархии.

\textbf{<<горячая>> перезагрузка} -- возможность обновить код или ресурсы приложения во время его выполнения без перезапуска.

\textbf{Тайл} -- графический элемент карты, организованный в сетку для построения игрового уровня.

\textbf{Ассет} -- ресурс проекта (спрайт, звук, шейдер и т. п.), импортируемый и используемый движком.

\textbf{Спрайт} -- растровое изображение, применяемое как игровой объект или элемент интерфейса.

\textbf{Текстура} -- графический ресурс, налагаемый на объекты сцены для их визуализации.

\textbf{Пакет} -- блок данных, передаваемый по сети между узлами приложения.

\textbf{Шейдер} -- программа для графического процессора, отвечающая за расчёт визуальных эффектов.

\textbf{P2P} -- архитектура <<равный-равному>>, где узлы сети обмениваются данными напрямую без центрального сервера.

\textbf{Клиент-серверная архитектура} -- модель взаимодействия, где клиенты отправляют запросы центральному серверу, а он отвечает и управляет состоянием.

\textbf{WER} -- метрика ошибок распознавания речи (Word Error Rate), процент отличий между эталонным и распознанным текстом.

\textbf{CPU} -- центральный процессор, основной вычислительный блок компьютера.

\textbf{GPU} -- графический процессор, специализированный блок для вычислений, связанных с визуализацией и шейдерами.

\textbf{HTTPS} -- протокол защищённой передачи гипертекста поверх TLS/SSL на базе HTTP.

\textbf{End-to-End} -- задержка или процесс, охватывающий весь путь данных от отправителя до конечного получателя без промежуточных буферов.

\textbf{MVP} -- минимально жизнеспособный продукт, версия проекта с минимальным набором функций для проверки идеи.

\textbf{Z-индекс} -- параметр порядка отрисовки объектов, определяющий, какие элементы отображаются поверх других.

\textbf{Плагин} -- модуль или расширение для программы или движка, добавляющее новую функциональность без внесения изменений в её ядро.

\textbf{Рендеринг} -- процесс вычисления и вывода на экран визуального представления сцены или графики, включающий расчёт освещения, текстур и шейдеров.

\textbf{Singleton} -- порождающий шаблон проектирования, гарантирующий создание единственного экземпляра класса и обеспечивающий к нему глобальный доступ.

\textbf{Сериализация} -- процесс преобразования объекта в последовательность байтов или иной формат для его сохранения, передачи по каналу связи или кэширования.

\textbf{Десериализация} -- процесс, при котором из ранее полученной последовательности байтов или иного формата восстанавливается исходный объект.