% !TEX root =..\main.tex
\section{ПОСТАНОВКА ЗАДАЧИ}

    \subsection{Актуальность} 
    Жанр RTS продолжает занимать важное место в индустрии, предлагая игрокам сложные сценарии управления и координации большого числа юнитов. 
    Традиционные средства управления (мышь и клавиатура) обеспечивают высокую точность, но требуют многократного переключения между меню и картой, 
    что замедляет реакцию. Внедрение голосовых команд способно упростить управление, повысить скорость отдачи приказов и сделать игру более 
    доступной для пользователей с особыми потребностями.

    Одновременно индустрия предъявляет всё более строгие требования к кроссплатформенности проектов и гибкости их архитектуры. Закрытые или узко 
    ориентированные игровые движки ограничены поддержкой платформ и набором встроенных инструментов, что существенно тормозит 
    развитие и масштабирование игры.

    В рамках курсовой работы был создан рабочий прототип сетевой RTS-игры на базе Citrus Engine с интегрированными модулями автоматического 
    распознавания речи и синтеза речи. Однако по мере расширения функционала выявились фундаментальные ограничения этой платформы.

    \subsection{Цель работы}
    Цель настоящей работы -- разработать и реализовать прототип сетевой RTS-игры на языке C\# с голосовым управлением, обеспечивающий:
    \begin{itemize}
        \item устойчивое многопользовательское взаимодействие;
        \item интеграцию локального ASR-модуля для распознавания голосовых команд;
        \item модуль TTS-синтеза для обратной голосовой связи;
        \item кроссплатформенный клиент на базе Godot Engine.
    \end{itemize}


    \subsection{Задачи}
    Для достижения поставленной цели необходимо решить следующие задачи:
    \begin{itemize}
    \item провести анализ исходного прототипа на Citrus Engine, выявить его технические и организационные ограничения;
    \item сформулировать требования к новому движку и обосновать выбор Godot;
    \item адаптировать и перенести логику игрового цикла и сетевого взаимодействия на Godot + C\#;
    \item реализовать клиентскую логику приёма голосовых команд, отрисовки игрового поля и воспроизведения аудиоответа.
    \end{itemize}

    \subsection{Недостатки исходного прототипа и обоснование перехода на Godot}
    Рабочий прототип на Citrus Engine включал клиент-серверную архитектуру, обработку пользовательского ввода, обмен игровым состоянием по 
    UDP, а также интеграцию ASR и TTS через выделенные модули. При всём этом были выявлены следующие ограничения:
    \begin{itemize}
    \item отсутствие кроссплатформенности, отсутствие экспорта в macOS, Linux и на мобильные устройства;
    \item фрагментарная и неполная документация, слабая поддержка сообщества;
    \item необходимость самостоятельной реализации многих низкоуровневых подсистем;
    \item громоздкая структура проекта, препятствующая быстрой навигации и автоматизации сборки.
    \end{itemize}

    Для устранения этих недостатков выбран Godot Engine, обладающий:
    \begin{itemize}
    \item полноценной кроссплатформенной поддержкой (Windows, Linux, macOS, iOS, Android, Web);
    \item развитой системой сцен и узлов, позволяющей строить игровую логику модульно;
    \item встроенной поддержкой C\# (Mono), упрощающей перенос существующего кода;
    \item активным сообществом и большим количеством готовых аддонов в Asset Library;
    \item возможностью «горячей» перезагрузки скриптов и гибкой настройкой проекта.
    \end{itemize}

    Переход на Godot обеспечивает сосредоточение основных усилий на развитии геймплейных механик и голосового интерфейса без отвлечения 
    на реализацию базовых движковых подсистем.