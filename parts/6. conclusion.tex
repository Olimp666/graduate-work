% !TEX root =..\main.tex
\section*{ЗАКЛЮЧЕНИЕ}\addcontentsline{toc}{section}{ЗАКЛЮЧЕНИЕ}

В результате выполненной работы получился полнофункциональный прототип сетевой RTS-игры, готовый к дальнейшей доработке и расширению. В отличие от 
первоначальной реализации на движке Citrus, где разработка упиралась в органичения движка, такие как отсутствие кроссплатформенности, 
слабую документацию и фрагментарный набор готовых инструментов, Godot позволил получить кроссплатформенный клиент со значительно расширенным функционалом 
благодаря его гибкой системе сцен и узлов, интуитивному редактору и встроенной поддержке C\#. Серверная часть сохраняет роль авторитетного хоста, обеспечивая
синхронность игровых состояний, а голосовой интерфейс на основе Vosk и gRPC работает стабильно и независимо от внешних облачных сервисов.

Godot доказал свою пригодность как для опытных, так и для начинающих разработчиков: понятные принципы композиции сцен и мгновенная «горячая» перезагрузка 
скриптов сокращают время обучения и позволяют сосредоточиться непосредственно на геймплейных механиках. 
Активное сообщество и подробная документация дают возможность быстро найти решение практически любой технической задачи, а визуальные инструменты для 
верстки HUD, настройки камер и работы с тайл-картами делают создание интерфейсов приятным и наглядным.

В будущем прототип легко может быть масштабирован: предусмотрены механизмы добавления новых типов юнитов, расширения правил боя, интеграции нейросетевых 
TTS-модулей и более сложных сценариев сетевого взаимодействия. Полученный результат может служить базой для учебных проектов, демо-версий коммерческих 
продуктов или открытого сообщества разработчиков. Благодаря открытому исходному коду и модульной архитектуре эта работа заложила надёжную платформу 
для любых дальнейших экспериментов и развития.