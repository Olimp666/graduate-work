%!LW recipe=latexmk (xelatex)
\section*{АННОТАЦИЯ}\addcontentsline{toc}{section}{АННОТАЦИЯ}

Выпускная квалификационная работа выполняет перенос и адаптацию логики сетевого прототипа RTS-игры с голосовым управлением с игрового движка Citrus, остававшегося на ранней стадии разработки и обладавшего ограниченной структурой, на кроссплатформенный Godot Engine с использованием языка программирования C\#. Проведён сравнительный анализ игровых движков и голосовых технологий, обоснован выбор Godot, локальной ASR-библиотеки Vosk.CSharp и .NET SpeechSynthesizer. Реализованы ключевые модули: клиент-серверное взаимодействие, gRPC-обработка голосовых команд, серверная логика боёв, HUD. Результаты подтверждают повышение архитектурной гибкости, производительности и кроссплатформенности прототипа.

\section*{ABSTRACT}

This thesis ported and adapted the logic of a networked RTS game prototype with voice control from the Citrus game engine — then at an early development stage with limited tooling — to the cross-platform Godot Engine using C\# programming language. A comparative analysis of game engines and voice technologies informed the choice of Godot, the Vosk.CSharp ASR library, and the .NET SpeechSynthesizer. Core modules were implemented: client-server communication, gRPC-based voice command processing, server-side battle logic, and HUD. The results demonstrate improved architectural flexibility, performance, and cross-platform support.